% !TEX TS-program = pdflatex
% !TEX encoding = UTF-8 Unicode

% This file is a template using the "beamer" package to create slides for a talk or presentation
% - Talk at a conference/colloquium.
% - Talk length is about 20min.
% - Style is ornate.

% MODIFIED by Jonathan Kew, 2008-07-06
% The header comments and encoding in this file were modified for inclusion with TeXworks.
% The content is otherwise unchanged from the original distributed with the beamer package.

\documentclass{beamer}

\usepackage[utf8]{inputenc}
\usepackage[T2A]{fontenc}     
\usepackage[russian]{babel}  

\usepackage{graphicx} 



% Copyright 2004 by Till Tantau <tantau@users.sourceforge.net>.
%
% In principle, this file can be redistributed and/or modified under
% the terms of the GNU General Public License, version 2.
%
% However, this file is supposed to be a template to be modified
% for your own needs. For this reason, if you use this file as a
% template and not specifically distribute it as part of a another
% package/program, I grant the extra permission to freely copy and
% modify this file as you see fit and even to delete this copyright
% notice. 


\mode<presentation>
{
  \usetheme{Madrid}
  % or ...

  \setbeamercovered{transparent}
  % or whatever (possibly just delete it)
}


%\usepackage[english]{babel}
% or whatever

%\usepackage[utf8]{inputenc}
% or whatever

%\usepackage{times}
%\usepackage[T1]{fontenc}
% Or whatever. Note that the encoding and the font should match. If T1
% does not look nice, try deleting the line with the fontenc.


\title[Программный комплекс Оптимус] % (optional, use only with long paper titles)
{Некоторые проекты компании ООО КАРСАР}

\subtitle
{2012 - 2025}

\author[В.~Г.~Бирюков] % (optional, use only with lots of authors)
{В.~Г.~Бирюков}
% - Give the names in the same order as the appear in the paper.
% - Use the \inst{?} command only if the authors have different
%   affiliation.

\institute[ООО КАРСАР] % (optional, but mostly needed)
{
  Ведущий инженер-программист\\
  Начальник отдела программирования ООО КАРСАР\\
  кандидат физико-математических наук
}
% - Use the \inst command only if there are several affiliations.
% - Keep it simple, no one is interested in your street address.

\date[2026] % (optional, should be abbreviation of conference name)
{Саратов, 2026}
% - Either use conference name or its abbreviation.
% - Not really informative to the audience, more for people (including
%   yourself) who are reading the slides online

\subject{Геофизическое исследование скважин (ГИС)}
% This is only inserted into the PDF information catalog. Can be left
% out. 



% If you have a file called "university-logo-filename.xxx", where xxx
% is a graphic format that can be processed by latex or pdflatex,
% resp., then you can add a logo as follows:

% \pgfdeclareimage[height=0.5cm]{university-logo}{university-logo-filename}
% \logo{\pgfuseimage{university-logo}}



% Delete this, if you do not want the table of contents to pop up at
% the beginning of each subsection:
%\AtBeginSubsection[]
%{
%\begin{frame}<beamer>{Оптимус}
%   \tableofcontents[currentsection,currentsubsection]
%  \end{frame}
%}


% If you wish to uncover everything in a step-wise fashion, uncomment
% the following command: 

%\beamerdefaultoverlayspecification{<+->}


\begin{document}

\begin{frame}
  \titlepage
\end{frame}

\begin{frame}{Методы ГИС}
  %\tableofcontents
  % You might wish to add the option [pausesections]
\begin{itemize}
\item \textbf{Стадартные методы ГИС}
	\begin{itemize}
	\item Инклинометрия
	\item Радиоактивные методы (активные/пассивные)
	\item Элктрические методы
	\item Акустические методы
	\end{itemize}
\item \textbf{Специальные методы ГИС}
	\begin{itemize}
	\item Электрический микросканер
	\item Акустический кроссдипольный каротаж
	\item Ядерно-магнитный каротаж
	\end{itemize}
\end{itemize}
\end{frame}


% Structuring a talk is a difficult task and the following structure
% may not be suitable. Here are some rules that apply for this
% solution: 

% - Exactly two or three sections (other than the summary).
% - At *most* three subsections per section.
% - Talk about 30s to 2min per frame. So there should be between about
%   15 and 30 frames, all told.

% - A conference audience is likely to know very little of what you
%   are going to talk about. So *simplify*!
% - In a 20min talk, getting the main ideas across is hard
%   enough. Leave out details, even if it means being less precise than
%   you think necessary.
% - If you omit details that are vital to the proof/implementation,
%   just say so once. Everybody will be happy with that.



%\section{Стандартные методы ГИС}

%\subsection{Радиоактивный каротаж (активный/пассивный)}
%\subsection{Электрические методы}
%\subsection{Акустические методы.}

\begin{frame}{Стандартные методы ГИС}
  % - A title should summarize the slide in an understandable fashion
  %   for anyone how does not follow everything on the slide itself.
\begin{itemize}
 \item \textbf{Радиоактивные методы ГИС}
	\begin{itemize}
	\item Спектрометрический гамма каротаж (СГК)
	\item Литоплотностной каротаж (ЛК)
	\item Импульсно-нейтронный каротаж (АИНК)
	\end{itemize}
\item \textbf{Электрические методы ГИС}
	\begin{itemize}
	\item Боковой каротаж (БК)
	\item Микробоковой каротаж (МБК)
	\item Многозондовый боковой каротаж (5БК)
	\item Многозондовый индукционный каротаж (5ИК, 6ИК) 
	\end{itemize}
\item \textbf{Акустические методы ГИС}
	\begin{itemize}
	\item Трёхэлементный зонд (АК73)
	\item Акустический телевизор
	\item Шумомер спектральный 
	\end{itemize}	
\end{itemize}
\end{frame}

\begin{frame}{Специальные методы ГИС}{Электрический микросканер}
{\small В состав прибора КарСар МС входят: модуль датчика натяжения и вращения, центратор, модуль ГК, модуль памяти и непрерывного инклинометра, позволяющего правильно ориентировать   в  пространстве,    полученную   информацию модуль сканера}
	\begin{figure}[h!] 
		\centering
		\includegraphics[width=0.8\textwidth, angle=0]{./Images/Схема МС3.jpg}
		\caption{Электрический микросканер МС-А}
	\end{figure} 
\end{frame}

\begin{frame}{Специальные методы ГИС}{Электрический микросканер}
\begin{itemize}
 \item \textbf{Регистратор}
	\begin{itemize}
	\item Калибровка радиусов (рычагов)
	\item Калибровка сопротивлений (электродов)
	\item Калибровка инклинометра
	\item Регистрация данных в память прибора
	\item Привязка данных к глубине (репроцессинг)
	\item Применение калибровочных коэффициентов
	\end{itemize}
\item \textbf{Интерпретатор (предварительная обработка данных)}
	\begin{itemize}
	\item 1
	\item  2
	\item 3
	\item 4
	\end{itemize}
\item \textbf{Интерпретатор (интерпретация данных)}
	\begin{itemize}
	\item Трёхэлементный зонд (АК73)
	\item Акустический телевизор
	\item Шумомер спектральный 
	\end{itemize}	
\end{itemize}
\end{frame}

\begin{frame}{Make Titles Informative.}
\end{frame}



\section{Our Results/Contribution}

\subsection{Main Results}

\begin{frame}{Make Titles Informative.}
\end{frame}

\begin{frame}{Make Titles Informative.}
\end{frame}

\begin{frame}{Make Titles Informative.}
\end{frame}


\subsection{Basic Ideas for Proofs/Implementation}

\begin{frame}{Make Titles Informative.}
\end{frame}

\begin{frame}{Make Titles Informative.}
\end{frame}

\begin{frame}{Make Titles Informative.}
\end{frame}



\section*{Summary}

\begin{frame}{Summary}

  % Keep the summary *very short*.
  \begin{itemize}
  \item
    The \alert{first main message} of your talk in one or two lines.
  \item
    The \alert{second main message} of your talk in one or two lines.
  \item
    Perhaps a \alert{third message}, but not more than that.
  \end{itemize}
  
  % The following outlook is optional.
  \vskip0pt plus.5fill
  \begin{itemize}
  \item
    Outlook
    \begin{itemize}
    \item
      Something you haven't solved.
    \item
      Something else you haven't solved.
    \end{itemize}
  \end{itemize}
\end{frame}



% All of the following is optional and typically not needed. 
\appendix
\section<presentation>*{\appendixname}
\subsection<presentation>*{For Further Reading}

\begin{frame}[allowframebreaks]
  \frametitle<presentation>{For Further Reading}
    
  \begin{thebibliography}{10}
    
  \beamertemplatebookbibitems
  % Start with overview books.

  \bibitem{Author1990}
    A.~Author.
    \newblock {\em Handbook of Everything}.
    \newblock Some Press, 1990.
 
    
  \beamertemplatearticlebibitems
  % Followed by interesting articles. Keep the list short. 

  \bibitem{Someone2000}
    S.~Someone.
    \newblock On this and that.
    \newblock {\em Journal of This and That}, 2(1):50--100,
    2000.
  \end{thebibliography}
\end{frame}

\end{document}


