% !TEX TS-program = pdflatex
% !TEX encoding = UTF-8 Unicode

% This file is a template using the "beamer" package to create slides for a talk or presentation
% - Talk at a conference/colloquium.
% - Talk length is about 20min.
% - Style is ornate.

% MODIFIED by Jonathan Kew, 2008-07-06
% The header comments and encoding in this file were modified for inclusion with TeXworks.
% The content is otherwise unchanged from the original distributed with the beamer package.

\documentclass{beamer}

\usepackage[utf8]{inputenc}
\usepackage[T2A]{fontenc}     
\usepackage[russian]{babel}  

\usepackage{graphicx} 


% Copyright 2004 by Till Tantau <tantau@users.sourceforge.net>.
%
% In principle, this file can be redistributed and/or modified under
% the terms of the GNU General Public License, version 2.
%
% However, this file is supposed to be a template to be modified
% for your own needs. For this reason, if you use this file as a
% template and not specifically distribute it as part of a another
% package/program, I grant the extra permission to freely copy and
% modify this file as you see fit and even to delete this copyright
% notice. 


\mode<presentation>
{
  \usetheme{Madrid}
  % or ...

  \setbeamercovered{transparent}
  % or whatever (possibly just delete it)
}


%\usepackage[english]{babel}
% or whatever

%\usepackage[utf8]{inputenc}
% or whatever

%\usepackage{times}
%\usepackage[T1]{fontenc}
% Or whatever. Note that the encoding and the font should match. If T1
% does not look nice, try deleting the line with the fontenc.


\title[Программный комплекс Оптимус] % (optional, use only with long paper titles)
{Некоторые проекты группы компаний КАРСАР}

\subtitle
{2012 - 2025}

\author[В.~Г.~Бирюков] % (optional, use only with lots of authors)
{В.~Г.~Бирюков}
% - Give the names in the same order as the appear in the paper.
% - Use the \inst{?} command only if the authors have different
%   affiliation.

\institute[ООО КАРСАР] % (optional, but mostly needed)
{
  Ведущий инженер-программист\\
  начальник отдела программирования ООО КАРСАР\\
  кандидат физико-математических наук
}
% - Use the \inst command only if there are several affiliations.
% - Keep it simple, no one is interested in your street address.

\date[2026] % (optional, should be abbreviation of conference name)
{Саратов, 2026}
% - Either use conference name or its abbreviation.
% - Not really informative to the audience, more for people (including
%   yourself) who are reading the slides online

\subject{Геофизическое исследование скважин (ГИС)}
% This is only inserted into the PDF information catalog. Can be left
% out. 



% If you have a file called "university-logo-filename.xxx", where xxx
% is a graphic format that can be processed by latex or pdflatex,
% resp., then you can add a logo as follows:

% \pgfdeclareimage[height=0.5cm]{university-logo}{university-logo-filename}
% \logo{\pgfuseimage{university-logo}}



% Delete this, if you do not want the table of contents to pop up at
% the beginning of each subsection:
%\AtBeginSubsection[]
%{
%\begin{frame}<beamer>{Оптимус}
%   \tableofcontents[currentsection,currentsubsection]
%  \end{frame}
%}


% If you wish to uncover everything in a step-wise fashion, uncomment
% the following command: 

%\beamerdefaultoverlayspecification{<+->}


\begin{document}

\begin{frame}
  \titlepage
\end{frame}

%\begin{frame}
%Презентация содержит некоторые проекты, в которых я принимал в период с 2012 по 2025 годы. Большая часть задач была %решена  непосредственно мной как математиком, инженером, программистом. Некоторые задачи я решал совместно с %коллегами. 
%\end{frame}

\begin{frame}{Методы ГИС}
  %\tableofcontents
  % You might wish to add the option [pausesections]
\begin{itemize}
\item \textbf{Стандартные методы ГИС}
	\begin{itemize}
	\item Инклинометрия
	\item Радиоактивные методы (активные/пассивные)
	\item Электрические методы
	\item Акустические методы
	\end{itemize}
\item \textbf{Специальные методы ГИС}
	\begin{itemize}
	\item Электрический микросканер
	\item Акустический кроссдипольный каротаж
	\item Ядерно-магнитный каротаж
	\end{itemize}
\end{itemize}
\end{frame}


% Structuring a talk is a difficult task and the following structure
% may not be suitable. Here are some rules that apply for this
% solution: 

% - Exactly two or three sections (other than the summary).
% - At *most* three subsections per section.
% - Talk about 30s to 2min per frame. So there should be between about
%   15 and 30 frames, all told.

% - A conference audience is likely to know very little of what you
%   are going to talk about. So *simplify*!
% - In a 20min talk, getting the main ideas across is hard
%   enough. Leave out details, even if it means being less precise than
%   you think necessary.
% - If you omit details that are vital to the proof/implementation,
%   just say so once. Everybody will be happy with that.



\begin{frame}{Стандартные методы ГИС}
  % - A title should summarize the slide in an understandable fashion
  %   for anyone how does not follow everything on the slide itself.
\begin{itemize}
 \item \textbf{Радиоактивные методы ГИС}
	\begin{itemize}
	\item Спектрометрический гамма каротаж (СГК)
	\item Литоплотностной каротаж (ЛК)
	\item Импульсно-нейтронный каротаж (АИНК)
	\end{itemize}
\item \textbf{Электрические методы ГИС}
	\begin{itemize}
	\item Боковой каротаж (БК)
	\item Микробоковой каротаж (МБК)
	\item Многозондовый боковой каротаж (5БК)
	\item Многозондовый индукционный каротаж (5ИК, 6ИК) 
	\end{itemize}
\item \textbf{Акустические методы ГИС}
	\begin{itemize}
	\item Трёхэлементный зонд (АК73)
	\item Акустический телевизор
	\item Шумомер спектральный 
	\end{itemize}	
\end{itemize}
\end{frame}

\begin{frame}{Специальные методы ГИС}{Электрический микросканер}
{
\centering
\huge
\textbf{ Электрический микросканер}
}
\end{frame}

\begin{frame}{Специальные методы ГИС}{Электрический микросканер}
Предназначен для получения изображения стенки скважины методом  электрических сопротивлений с целью определения:
\begin{itemize}
	\item угла наклона и азимута падения пластов; 
	\item расположения трещин, их наклона и направления;
	\item параметров трещин;
	\item структуры осадочных пород;
	\item исследования тонкослоистых структур; 
	\item механических свойств ствола скважин;
	\item профиля скважины;
	\item разделение пористости на первичную и вторичную компоненты.
	\end{itemize}
\end{frame}

\begin{frame}{Специальные методы ГИС}{Электрический микросканер}
Данные электрического микросканера могут быть использованы для определения: 
\begin{itemize}
	\item обстановки осадконакопления; 
	\item тектонических условий формирования горных пород;
	\item их структурных и текстурных особенностей; 
	\item для стратиграфической корреляции;
	\item для уточнения геомеханической модели среды;
	\item в качестве дополнения к данным, получаемым в результате исследования керна.
	\end{itemize}
\end{frame}

\begin{frame}{Специальные методы ГИС}{Электрический микросканер}
{\small В состав прибора КарСар МС входят: модуль датчика натяжения и вращения, центратор, модуль ГК, модуль памяти и непрерывного инклинометра, позволяющего правильно ориентировать   в  пространстве,    полученную   информацию, модуль сканера}
	\begin{figure}[h!] 
		\centering
		\includegraphics[width=0.8\textwidth, angle=0]{./Images/microscanner_view.jpg}
		\caption{Электрический микросканер МС-А}
	\end{figure} 
\end{frame}

\begin{frame}{Специальные методы ГИС}{Электрический микросканер}
	\begin{figure}[h!] 
		\centering
		\includegraphics[width=0.8 \textwidth, angle=0]{./Images/microscanner_pad.jpg}
		\caption{Башмак электрического микросканера}
	\end{figure} 
\end{frame}

\begin{frame}{Специальные методы ГИС}{Электрический микросканер}
\begin{itemize}
 \item \textbf{Регистратор}
	\begin{itemize}
	\item Калибровка радиусов (рычагов)
	\item Калибровка потенциала
	\item Калибровка сопротивлений (электродов)
	\item Калибровка инклинометра
	\item Регистрация данных в память прибора
	\item Привязка данных к глубине (репроцессинг)
	\item Применение калибровочных коэффициентов
	\end{itemize}
\end{itemize}
\end{frame}

\begin{frame}{Специальные методы ГИС}{Электрический микросканер}
	\begin{figure}[h!] 
		\centering
		\includegraphics[width=0.9 \textwidth, angle=0]{./Images/microscanner_calibration.jpg}
		\caption{Калибровка электрического микросканера}
	\end{figure} 
\end{frame}

\begin{frame}{Специальные методы ГИС}{Электрический микросканер}
	\begin{figure}[h!] 
		\centering
		\includegraphics[width=0.7 \textwidth, angle=0]{./Images/microscanner_reprocessing.jpg}
		\caption{Репроцессинг электрического микросканера}
	\end{figure} 
\end{frame}

\begin{frame}{Специальные методы ГИС}{Электрический микросканер}
\begin{itemize}
\item \textbf{Интерпретатор (предварительная обработка данных)}
	\begin{itemize}
	\item Коррекция за неравномерное движение прибора
	\item Коррекция за отсутствие данных на отдельных электродах
	\item Коррекция за эксцентриситет
	\item Эквализация данных
	\item Сцепление и ориентация башмаков
	\item Построение статического и динамического имиджа
	\item Фильтрация имиджа 
	\item Калибровка имиджа по данным МБК
	\item Восстановление имиджа
	\end{itemize}
\item \textbf{Интерпретатор (интерпретация данных)}
	\begin{itemize}
	\item Выделение геологических объектов
	\item Классификация геологических объектов
	\item Расчет вторичной пористости
	\item Анализ вывалов скважины
	\end{itemize}	
\end{itemize}
\end{frame}

\begin{frame}{Специальные методы ГИС}{Электрический микросканер}
\begin{figure}[h!] 
		\centering
		\includegraphics[width=0.8 \textwidth, angle=0]{./Images/microscanner_image1.jpg}
		\caption{Пример имиджа электрического микросканера}
	\end{figure} 
\end{frame}

\begin{frame}{Специальные методы ГИС}{Электрический микросканер}
\begin{figure}[h!] 
		\centering
		\includegraphics[width=0.8 \textwidth, angle=0]{./Images/microscanner_image2.jpg}
		\caption{Пример восстановленного имиджа}
	\end{figure} 
\end{frame}

\begin{frame}{Специальные методы ГИС}{Электрический микросканер}
\begin{figure}[h!] 
		\centering
		\includegraphics[width=0.8 \textwidth, angle=0]{./Images/microscanner_image3.jpg}
		\caption{Пример планшета}
	\end{figure} 
\end{frame}

\begin{frame}{Специальные методы ГИС}{Кроссдипольный акустический каротаж}
\centering
\huge
\textbf{Кроссдипольный акустический каротаж}
\end{frame}

\begin{frame}{Специальные методы ГИС}{Кроссдипольный акустический каротаж}
\textbf{Решаемые задачи:}
\begin{itemize}
	\item Литологическое расчленение разреза
	\item Определение коэффициента и типа пористости пород
	\item Выделение проницаемых пластов
	\item Расчёт модулей упругости горных пород
	\item Расчет направления анизотропии по данным кросс-дипольного зонда
\end{itemize}
\end{frame}

\begin{frame}{Специальные методы ГИС}{Кроссдипольный акустический каротаж}
\textbf{Основные характеритики:}
\begin{itemize}
	\item Восемь приемных станций обеспечивают максимальную точность расчетов по сравнению с  аппаратурой двухканальной, типа АВАК, МАК и т.п.
	\item Низкая частота работы дипольного излучателя (1.8 кГц) позволяет надежно определять параметры поперечных волн без коррекции за дисперсию скоростей
	\item Диаметр прибора 73мм позволяет работать в скважинах малого диаметра (от 95мм) и через буровой инструмент
	\item Работа на трёхжильном геофизическом кабеле	
\end{itemize}
\end{frame}

\begin{frame}{Специальные методы ГИС}{Кроссдипольный акустический каротаж}
В состав прибора входят:
\begin{itemize}
	\item Модуль приемников: восемь 4-х канальных станций обеспечивают регистрацию волновых пакетов
	\item Модуль излучателей: монопольный и два кросс-дипольных излучателя
\end{itemize}
	\begin{figure}[h!] 
		\centering
		\includegraphics[width=1.0 \textwidth, angle=0]{./Images/crossdipole_view.jpg}
		\caption{Общий вид прибора КарСар 8АД }
	\end{figure} 
\end{frame}

\begin{frame}{Специальные методы ГИС}{Кроссдипольный акустический каротаж}
\textbf{Этапы обработки:}
\begin{itemize}
	\item Репроцессинг данных из памяти прибора	
	\item Привязка данных к глубине
	\item Вычисление интервального времени продольной волны, поперечной волны и волны Стоунли	
	\item Получение скоростей быстрой и медленной поперечной волны
	\item Определение акустической анизотропии и ее азимутальной ориентации
	\item Определение скорости  быстрой  и  медленной  поперечных  волн и ориентации их  векторов
\end{itemize}
\end{frame}

\begin{frame}{Специальные методы ГИС}{Кроссдипольный акустический каротаж}
	\begin{figure}[h!] 
		\centering
		\includegraphics[width=1.0 \textwidth, angle=0]{./Images/crossdipole_summary.jpg}
		\caption{Общий вид планшета с результатами обработки }
	\end{figure} 
\end{frame}

\begin{frame}{Специальные методы ГИС}{Кроссдипольный акустический каротаж}
	\begin{figure}[h!] 
		\centering
		\includegraphics[width=0.4 \textwidth, angle=0]{./Images/crossdipole_stc.jpg}
		\caption{ Результат обработки STC }
	\end{figure} 
\end{frame}

\begin{frame}{Специальные методы ГИС}{Кроссдипольный акустический каротаж}
	\begin{figure}[h!] 
		\centering
		\includegraphics[width=0.5 \textwidth, angle=0]{./Images/crossdipole_anisothropy.jpg}
		\caption{ Направление анизотропии }
	\end{figure} 
\end{frame}

\begin{frame}{Специальные методы ГИС}{Кроссдипольный акустический каротаж}
	\begin{figure}[h!] 
		\centering
		\includegraphics[width=1.0 \textwidth, angle=0]{./Images/crossdipole_disperssion.jpg}
		\caption{ Карта дисперсии }
	\end{figure} 
\end{frame}

\begin{frame}{Специальные методы ГИС}{Ядерно-магнитный каротаж}
\centering
\huge
\textbf{Ядерно-магнитный каротаж}
\end{frame}

\begin{frame}{Специальные методы ГИС}{Ядерно-магнитный каротаж}
\textbf{Решаемые задачи:}
\begin{itemize}
	\item Общая пористость
	\item Эффективная пористость
	\item Связанный флюид
	\item Подвижный флюид
	\item Вода в глинах
	\item Проницаемость
	\item Типизация флюида (анализ карты диффузии)
\end{itemize}
\end{frame}

\begin{frame}{Специальные методы ГИС}{Ядерно-магнитный каротаж}
\begin{itemize}
 \item \textbf{Регистратор}
	\begin{itemize}
	\item Калибровка резонансных частот
	\item Калибровка передатчика
	\item Калибровка приёмника
	\item Верификация приёмника
	\item Регистрация данных
	\end{itemize}
\end{itemize}
\end{frame}

\begin{frame}{Специальные методы ГИС}{Ядерно-магнитный каротаж}
\begin{figure}[h!] 
		\centering
		\includegraphics[width=0.7 \textwidth, angle=0]{./Images/nmr_frequency_sweep.jpg}
		\caption{ Калибровка резонансных частот }
	\end{figure} 
\end{frame}

\begin{frame}{Специальные методы ГИС}{Ядерно-магнитный каротаж}
\begin{figure}[h!] 
		\centering
		\includegraphics[width=0.7 \textwidth, angle=0]{./Images/nmr_transmitter_calibration.jpg}
		\caption{ Калибровка передатчика }
	\end{figure} 
\end{frame}

\begin{frame}{Специальные методы ГИС}{Ядерно-магнитный каротаж}
\begin{figure}[h!] 
		\centering
		\includegraphics[width=0.7 \textwidth, angle=0]{./Images/nmr_receiver_calibration.jpg}
		\caption{ Калибровка приёмника }
	\end{figure} 
\end{frame}

\begin{frame}{Специальные методы ГИС}{Электрический микросканер}
\begin{itemize}
\item \textbf{Интерпретатор (предварительная обработка данных)}
	\begin{itemize}
	\item Применение калибровочных коэффициентов 
	\item Устранение выбросов (despiking)
	\item PAP-коррекция 
	\item Корреция фазы	
	\item Комбинирование разночастотных данных
	\item Усреднение
	\end{itemize}
\item \textbf{Интерпретатор (интерпретация данных)}
	\begin{itemize}
	\item Построение $T_{2}$-спектра
	\item Расчет пористостей: общая, глинистая, подвижный флюид, эффективная
	\item Построение карты диффузии
	\item Типизация флюида
	\item Остаточное водо- и нефтеначыщение
	\end{itemize}	
\end{itemize}
\end{frame}

\begin{frame}{Специальные методы ГИС}{Ядерно-магнитный каротаж}
\begin{figure}[h!] 
		\centering
		\includegraphics[width=0.85 \textwidth, angle=0]{./Images/nmr_explorer.jpg}
		\caption{ Эхо-сигналы }
	\end{figure} 
\end{frame}

\begin{frame}{Специальные методы ГИС}{Ядерно-магнитный каротаж}
\begin{figure}[h!] 
		\centering
		\includegraphics[width=0.7 \textwidth, angle=0]{./Images/nmr_T2Viewer.jpg}
		\caption{ Просмотр $T_{2}$-спектров }
	\end{figure} 
\end{frame}

\begin{frame}{Специальные методы ГИС}{Ядерно-магнитный каротаж}
\begin{figure}[h!] 
		\centering
		\includegraphics[width=0.5 \textwidth, angle=0]{./Images/nmr_T2.jpg}
		\caption{ $T_{2}$-спектры и биновая пористость }
	\end{figure} 
\end{frame}

\begin{frame}{Специальные методы ГИС}{Ядерно-магнитный каротаж}
\begin{figure}[h!] 
		\centering
		\includegraphics[width=0.7 \textwidth, angle=0]{./Images/nmr_diffusivity.jpg}
		\caption{ Анализ $D$-$T_{2}$ - карты }
	\end{figure} 
\end{frame}



\end{document}


